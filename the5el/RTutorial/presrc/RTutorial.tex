\documentclass{beamer}

\mode<presentation>
{
  \usetheme{Copenhagen}
  \usecolortheme{beaver}
  \setbeamercovered{transparent}
  \setbeamertemplate{items}[ball]
  \setbeamertemplate{theorems}[numbered]
  \setbeamertemplate{footline}[frame number]
}

\usepackage{beamerthemesplit}
\usepackage{graphics}
\usepackage{graphicx}
\usepackage{hyperref}
\usepackage{listings}

\title{ Data Analysis and Visualisation using R }
\author{ Vinayak Hegde }
\date{ July 11, 2013 }

\AtBeginSection[]  % "Beamer, do the following at the start of every section"
{
\begin{frame}<beamer> 
\frametitle{Outline} % make a frame titled "Outline"
\tableofcontents[currentsection]  % show TOC and highlight current section
\end{frame}
}
\begin{document}

\frame {
  \titlepage
}

\section*{Outline}
\frame
{ \frametitle{Outline of Topics}
  \tableofcontents
}

\section{ Introduction }
\frame {
  \frametitle { What is R ? }
  \begin{block}	{Wikipedia} R is a free software programming language and a software environment for statistical computing and graphics. The R language is widely used among statisticians and data miners for developing statistical software and data analysis.
  \end{block}
}

\frame {
  \frametitle { Why use R ? }
  \begin{itemize}
    \item Designed and optimised for data processing
    \item Lots of modules
    \item State of the art graphics
    \item Free as in freedom/beer
    \item Helpful community
    \item Very flexible and good integration
  \end{itemize}
}
 
\frame {
  \frametitle { R Studio Installation }
  \begin{itemize}
    \item Go to \href{http://www.rstudio.com/ide/download/}{{\alert {RStudio website}}}
    \item Download the server/desktop version
    \item For server - Open the browser and go to http://127.0.0.1:8787
    \item For desktop - Click on the shortcut and you are ready to go
  \end{itemize}
}

\frame {
  \frametitle { R Packages Installation }
  \alert{install.packages('packageName')} 
  Select the closest mirror and install in your local directory
 
   \begin{itemize}
      \item plyr
      \item ggplot2
      \item gdata
      \item lattice
   \end{itemize}
}

\frame {
  \frametitle { Basics - Starting off and getting help }
  \begin{itemize}
    \item Starting the interpreter
    \item Getting online help (?)
    \item Searching for help (??)
    \item loading a library 
  \end{itemize}
}

\frame {
  \frametitle { Basics - Using inbuilt function }
    \item str
    \item summary
    \item head
    \item View
    \item Assignment <-
}

\section { Data Structures }
\frame {
   \frametitle { Data Frames }
} 

\section { Reading Data }
\frame {
   \frametitle { Multiple Sources }
   \begin{itemize}
      \item Excel files
      \item web pages
      \item csv
      \item databases
   \end{itemize}
}
\section { Transforming Data }
\frame {
   \frametitle { plyr }
}

\section { Packages and Datasets }
\frame {
   \frametitle { Packages }
   \begin{itemize}
      \item plyr
      \item ggplot2
      \item reshape2
      \item lattice
   \end{itemize}
}

\frame {
   \frametitle { Datasets }
   \begin{itemize}
      \item iris
      \item mtcars
      \item diamonds
      \item HairEyeColor
   \end{itemize}
}

\section { Visualisation }
\frame {
   \frametitle { Viz Packages }
   \begin{itemize}
      \item boxplot 
      \item ggplot2
      \item plot
      \item lattice
   \end{itemize}
}

\section { Webapps }
\frame {
   \frametitle { Shiny }
}

\section { Integration with other Systems }
\frame {
   \frametitle { Packages }
   \begin{itemize}
      \item hadoop 
      \item c++
      \item javascript
   \end{itemize}
}

\end{document}
